\chapter{Einleitung}
In unserer Arbeit untersuchen wir die Möglichkeiten welche die Clouddatenbank Google Fusion Tables bietet. Es sollen einige Prototypen für verschiedenste Anwendungsfälle im GIS-Bereich erstellt werden, welche das Potential der Datenbank aufzeigen.

Die Aufgabenstellung stammt von der GEOINFO AG, welche massgeschneiderte GIS-
Softwarelösungen für ihre Kunden entwickelt.

Für solche Unternehmen wird es nach und nach schwieriger sich auf dem Markt zu beweisen, da bereits viele cloudbasierte GIS-Lösungen sehr günstig oder gar kostenlos erhältlich sind. Durch die Prototypen soll ersichtlich gemacht werden, welche Anwendungsfälle von bestehenden proprietären GIS-Sytemen bereits mit Google Fusion Tables realisierbar wären und welchen Aufwand dies darstellen würde.

\section{Problemstellung}
Die Arbeit umfasst zwei Themenbereiche. Einen theoretischen Teil in dem untersucht wird, inwiefern Google Fusion Tables eine Konkurrenz für bestehende proprietäre GIS-Lösungen darstellt. Als zweiten Teil sollen verschiedene Anwendungsfälle mit Google Fusion Tables als Prototypen nachgebaut werden, wobei einer davon als mobile Webapplikation ausprogrammiert werden soll.

\subsection{Potential von Google Fusion Tables}
In einem theoretischen Teil sollen Erkenntnisse gewonnen werden, inwiefern sich bestehende GIS-Lösungen mit Google Fusion Tables ablösen lassen.

Dies soll am Beispiel eines Migrationsszenarios festgestellt werden. Die bestehenden GIS-Daten sollen möglichst einfach Exportiert und anschliessend in eine Google Fusion Tables-Datenbank importiert werden. Dafür werden die geeigneten Formate zur Übertragung der Daten evaluiert.

Da sich die Datenstrukturen von bestehenden Lösungen stark unterscheiden können, ist es nicht das Ziel eine Schritt-für-Schritt Anleitung zu erstellen. Primär geht es darum verschiedene Migrationslösungen zu entwickeln und untereinander zu vergleichen.

Dadurch soll das gegenwärtige Potential von Google Fusion Tables abgeschätzt werden. Diese Erkenntnis soll GIS-Lösungsprovidern einen Überblick verschaffen, inwiefern Google Fusion Tables bereits als einen möglichen Konkurrenten angesehen werden muss.

\section{Ziele}
We hope you will enjoy using this release as much as we enjoyed creating it. If you have comments, suggestions or wish to report an issue you are experiencing - contact us at: \emph{http://gummi.midnightcoding.org}.

\section{Rahmenbedingung}

\section{Vorgehen}

\section{Aufbau der Arbeit}

\section{Stand der Technik}

\section{Bewertung (Evaluation)}

\section{Vision/Umsetzungskonzept}

\section{Resultate der Arbeit}

\section{Schlussfolgerungen}

\section{Ausblick}