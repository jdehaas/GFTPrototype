\chapter{Einleitung}
Einleitung

\section{Problemstellung}

Dies ist der erste Unterpunkt im ersten Kapitel.

Strukturiert mit dem {\LaTeX} Systems, \emph{rubber}\footnote{https://launchpad.net/rubber/} \& \emph{latexmk}\footnote{http://www.phys.psu.edu/{\textasciitilde}collins/software/latexmk-jcc/}. 

Added for your viewing convenience is a continuous preview mode for the PDF. This mode is enabled by default, but can also be disabled through the \emph{(View $\rightarrow$ Page layout in preview)} menu. Complementary to this feature is SyncTeX integration, which allows you to synchronize the position in your editor with the PDF preview. 

\section{Ziele}
We hope you will enjoy using this release with your \gls{computer} as much as we enjoyed creating it. If you have comments, suggestions or wish to report an issue you are experiencing - contact us at: \emph{http://gummi.midnightcoding.org}.

\section{Rahmenbedingung}

\section{Vorgehen}

\section{Aufbau der Arbeit}

\section{Stand der Technik}

\section{Bewertung (Evaluation)}

\section{Vision/Umsetzungskonzept}

\section{Resultate der Arbeit}

\section{Schlussfolgerungen}

\section{Ausblick}