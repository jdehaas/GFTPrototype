\section{Sprint 1}

Der erste Sprint stand ganz im Zeichen des Einarbeitens. Es ging uns darum mit dem Google Fusion Tables \gls{API} bekannt zu werden. Daneben musste noch unsere Infrastruktur eingerichtet werden. Dazu zählt das Repository, unser Projektmanagement-Tool und der Build-Server.

Alle Informationen zum Sprint 1 sind auch in unserem Wiki zu finden:
\url{http://redmine.rdmr.ch/redmine/projects/gftprototype/wiki/Sprint_1}

\subsubsection{Hauptaufgaben / Fokussierung im Sprint}

\begin{itemize}
	\item Aufsetzen der Infrastruktur (Repository, Projektmanagement, Entwicklungsumgebung, Server)
	\item Einarbeitung in die Thematik
	\begin{itemize}
		\item \gls{GIS}
		\item Google Fusion Tables (GFT)
		\item allenfalls weitere \gls{API}s
	\end{itemize}
	\item Erarbeitung eines ersten Roundtrips um Daten von und zu Google Fusion Tables zu schicken/empfangen
	\item Projektsetup
	\begin{itemize}
		\item GitHub\footnote{\url{http://www.github.com}} als Repository
		\item LaTex\footnote{\url{http://www.latex-project.org}} für Dokumentation
		\item Jenkins \footnote{\url{http://jenkins-ci.org}} für Continuous Integration (CI)
		\item Redmine \footnote{\url{http://www.redmine.org}} für Projektmanagement/Wiki/Bugtracker
		\item Scrum als Methodik
	\end{itemize}
\end{itemize}

\subsubsection{Ziele}
\begin{itemize}
	\item Google Fusion Table \gls{API} kennen lernen, Potential abschätzen können
	\item Erster Prototyp mit GFT bauen (Roundtrip mit \gls{CRUD}-Operationen)
\end{itemize}

\subsubsection{Abgabe / Deliverables}
Wir sind im ersten Sprint gut vorangekommen und konnten mit zahlreichen aufeiander aufbauenden Beispielen lernen wie das \gls{API} funktioniert und welche Möglichkeiten es bietet: Abfragen erstellen, Zugriff via \gls{API}, Zugriff via Google Maps Layer (FusionTablesLayer).

\begin{itemize}
	\item Repository, Build-Server und Projekmanagement-Tool aufgesetzt (siehe Kapitel \ref{infrastruktur})
	\item Lauffähiger Prototyp mit Unit-Test für \gls{CRUD}-Operationen
	\item Zahlreiche Beispiele um die Funktionsweise des \gls{API}s zu testen
\end{itemize}

\subsubsection{Probleme}
Es ist uns nicht gelungen den Roundtrip zu erstellen, d.h. Daten vom Benutzer in Fusion Tables zu speichern und diese wieder abzufragen. Wie sich herausgestellt hat, müssen schreibende Zugriffe autorisiert sein. Dazu empfiehlt Google \gls{OAuth} zu benutzen. Diese Thematik war schlicht zu gross um im ersten Sprint anzuschauen. Die schreibenden Zugriffe sind deshalb Ziel für den 2. Sprint geworden. 