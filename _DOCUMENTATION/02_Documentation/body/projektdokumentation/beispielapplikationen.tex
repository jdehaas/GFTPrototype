\chapter{Beispielapplikationen}
\label{beispielapplikationen}
Um die verschiedenen Features der Google Fusion Tables kennenzulernen, erstellten wir zu Beginn der Arbeit einige kleine Beispielapplikationen, welche diese verwenden.

Eine Übersicht über alle erstellen Beispiele findet man auf folgender Übersichtsseite: \url{http://gft.rdmr.ch/}

\section{Daten selektieren}
\begin{tabular}{p{0.2\twocelltabwidth}p{0.8\twocelltabwidth}}
\textbf{URL:} & \url{http://gft.rdmr.ch/examples/js/data-print/} \\ 
\textbf{Beschreibung:} & Daten mittels gftlib-js von FusionTable selektieren und textuell ausgeben. \\ 
\end{tabular} 

\section{Daten selektieren mit ortsbezogener Einschränkung}
\begin{tabular}{p{0.2\twocelltabwidth}p{0.8\twocelltabwidth}}
\textbf{URL:} & \url{http://gft.rdmr.ch/examples/js/spatialquery-condition/} \\ 
\textbf{Beschreibung:} & Daten mittels gftlib-js von FusionTable selektieren und textuell ausgeben. Die Daten werden aber mit dem Spatial-Query \inlinecode{ST\_INTERSECTS} (siehe Abschnitt \ref{sqlapi-spatialqueries}) eingeschränkt. \\ 
\end{tabular} 

\section{Daten selektieren mit ortsbezogener Sortierung}
\begin{tabular}{p{0.2\twocelltabwidth}p{0.8\twocelltabwidth}}
\textbf{URL:} & \url{http://gft.rdmr.ch/examples/js/spatialquery-order/} \\ 
\textbf{Beschreibung:} & Daten mittels gftlib-js von FusionTable selektieren und textuell ausgeben. Die zurückgelieferten Daten werden bei der Abfrage mit dem Spatial-Query \inlinecode{ST\_DISTANCE} (siehe Abschnitt \ref{sqlapi-spatialqueries}) sortiert. \\ 
\end{tabular} 

\section{Google Maps: Daten auf Karte anzeigen}
\begin{tabular}{p{0.2\twocelltabwidth}p{0.8\twocelltabwidth}}
\textbf{URL:} & \url{http://gft.rdmr.ch/examples/js/gmap-rawdata/} \\ 
\textbf{Beschreibung:} & Daten mittels gftlib-js von FusionTable selektieren und auf Karte anzeigen. \\ 
\end{tabular} 

\section{Google Maps: Daten auf Karte anzeigen mit manueller Geocodierung}
\begin{tabular}{p{0.2\twocelltabwidth}p{0.8\twocelltabwidth}}
\textbf{URL:} & \url{http://gft.rdmr.ch/examples/js/gmap-geocoding/} \\ 
\textbf{Beschreibung:} & Daten mittels gftlib-js von FusionTable selektieren und auf Karte anzeigen. Um die Markierungen zu positionieren wird der Geocodierungs-Service des Google Maps API verwendet.  \\ 
\end{tabular} 

\section{Google Maps: Fusion Table-Ebene}
\begin{tabular}{p{0.2\twocelltabwidth}p{0.8\twocelltabwidth}}
\textbf{URL:} & \url{http://gft.rdmr.ch/examples/js/gmap-fusiontableslayer/} \\ 
\textbf{Beschreibung:} & Anzeigen der Daten aus zwei FusionTables via FusionTablesLayer.  \\ 
\end{tabular} 

\section{Google Maps: Fusion Table-Ebene Stile}
\begin{tabular}{p{0.2\twocelltabwidth}p{0.8\twocelltabwidth}}
\textbf{URL:} & \url{http://gft.rdmr.ch/examples/js/gmap-fusiontableslayer-clickstyle/} \\ 
\textbf{Beschreibung:} & Anzeigen der Daten aus einer FusionTable via FusionTablesLayer mit manuell konfiguriertem Stil (siehe Abschnitt \ref{fusiontableslayer-styles}). Sobald auf eine angezeigte Fläche geklickt wird, ändert sich deren Farbe. \\ 
\end{tabular} 

\section{Google Maps: Dynamische Fusion Table-Ebene}
\begin{tabular}{p{0.2\twocelltabwidth}p{0.8\twocelltabwidth}}
\textbf{URL:} & \url{http://gft.rdmr.ch/examples/js/gmap-dynamic-fusiontableslayer/} \\ 
\textbf{Beschreibung:} & Anzeigen der Daten aus zwei FusionTables via FusionTablesLayer. Per Slider lassen sich zusätzlich die zurückgegebenen Daten nach der Anzahl an Einwohnern einschränken.  \\ 
\end{tabular} 

\section{Google Maps: Fusion Table-Ebene mit ortsbezogener Einschränkung}
\begin{tabular}{p{0.2\twocelltabwidth}p{0.8\twocelltabwidth}}
\textbf{URL:} & \url{http://gft.rdmr.ch/examples/js/gmap-spatialquery/} \\ 
\textbf{Beschreibung:} & Anzeigen der Daten aus einer FusionTable via FusionTablesLayer. Es werden nur diejenigen Daten angezeigt, welche im Radius der auf der Karte positionierten Markierung liegen. Die Markierung lässt sich per Drag{\&}Drop auf der Karte verschieben. Zusätzlich lässt sich der Radius per Slider definieren. \\ 
\end{tabular} 

\section{Google Charts: Daten mit Diagrammen visualisieren}
\begin{tabular}{p{0.2\twocelltabwidth}p{0.8\twocelltabwidth}}
\textbf{URL:} & \url{http://gft.rdmr.ch/examples/js/gchart-fusiontable/} \\ 
\textbf{Beschreibung:} & Visualisieren der Daten aus einer FusionTable mit Diagrammen. Dazu wird das Google Chart Tool API verwendet. \\ 
\end{tabular} 

\section{Einfügen von Daten}
\begin{tabular}{p{0.2\twocelltabwidth}p{0.8\twocelltabwidth}}
\textbf{URL:} & \url{http://gft.rdmr.ch/examples/js/oauth-login/} \\ 
\textbf{Beschreibung:} & Einfügen von Daten in eine FusionTable. Dazu muss zuerst ein Zugriffs-Token via OAuth angefordert werden.

\textit{Hinweis: Dieses Beispiel funktioniert nur mit einem Google Account.} \\ 
\end{tabular} 

\section{Verwendung einer Fusion Table-Ebene in Sencha Touch 2}
\begin{tabular}{p{0.2\twocelltabwidth}p{0.8\twocelltabwidth}}
\textbf{URL:} & \url{http://gft.rdmr.ch/examples/js/senchatouch-fusiontableslayer/} \\ 
\textbf{Beschreibung:} & Anzeige einer FusionTableLayer in einer Sencha Touch 2 Applikation. \\ 
\end{tabular} 