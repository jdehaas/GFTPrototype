\chapter{Use Case 1: WorldData Explorer}

\section{Einführung}
Im ersten Use Case geht es hauptsächlich um die Anzeige grosser Datenmengen auf der Karte. Dazu importieren wir bestehende Datenbestände in die Google Fusion Tables und visualisieren diese mittels Google Maps API auf der Karte.

\subsection{Ziel}
Es sollen länderspezifische Daten auf einer Weltkarte angezeigt werden. Diese Daten lassen sich dann mittels einer Zeitachse pro Jahr visualisieren. Daraus können beispielsweise Zusammenhänge von verschiedenen Datenkategorien gefunden werden.

Um die Daten pro Land zu visualisieren, werden zuerst die Landesgrenzen als Geometrie-Datensätze in eine separate Fusion Tabelle importiert. In eine andere Tabelle werden dann die ganzen Daten importiert unterteilt nach Land und Jahr.

\subsubsection{ERD}


\subsection{Datenqullen}

\section{Anforderungsspezifikation}

\section{Analyse}

\section{Design}

\section{Implementation}

\section{Test}

\section{Resultate}

\section{Weiterentwicklung}

\section{Benutzerdokumentation}