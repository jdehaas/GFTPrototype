\subsection{Sprint 3}

Im dritten Sprint war das Hauptziel die Umsetzung des zweiten Use Cases \emph{FixMyStreet} als mobile WebApp. Dies beinhaltete unter anderem auch die Anbindung des Sencha Touch 2 Frameworks an die Google Fusion Table.
Zusätzlich wollten wir noch einen Converter-Build erstellen, mit welchem es möglich ist verschiedenste \gls{GIS}-Formate in eine Google Fusion Table zu importieren oder in andere Formate zu konvertieren.

Alle Informationen zum Sprint 3 sind auch in unserem Wiki zu finden:
\url{http://redmine.rdmr.ch/redmine/projects/gftprototype/wiki/Sprint_3}

\subsubsection{Hauptaufgaben / Fokussierung im Sprint}
\begin{itemize}
	\item Zweiter Use Case \emph{FixMyStreet} implementieren
	\item Converter-Build erstellen
\end{itemize}

\subsubsection{Ziele}
\begin{itemize}
	\item Lauffähige App für \emph{FixMyStreet} Use Case
	\begin{itemize}
		\item Eintragen von \emph{Defekten}
		\item Anzeigen der Defekte auf Karte
		\item Daten von GFT lesen / in GFT schreiben
	\end{itemize}
\end{itemize}

\subsubsection{Abgabe / Deliverables}
\begin{itemize}
	\item Lauffähige Version der \emph{FixMyStreet}-WebApp
	\item Converter-Build (GIS Formate konvertieren bzw. in GFT importieren)
\end{itemize}

Am Ende dieses Sprints hatten wir eine lauffähige Version der \emph{FixMyStreet}-WebApp (Kapitel \ref{fixmystreet}). Diese beinhaltete die Anbindung einer Google Fusion Table als Datenbank. Es sind aber noch einige kleine Bugs vorhanden, welche im nächsten Sprint korrigiert werden müssen.

\subsubsection{Probleme}
In der App sind noch einige Bugs in Bezug auf das Lesen und Schreiben der Daten aus einer Fusion Table vorhanden. Es scheint als gäbe es Probleme mit der internen (in App) und externen ID (in Fusion Talbe) der Datensätze. Zudem wurde bei verschiedenen Tests noch einige Usability-Probleme festgestellt, welche noch behoben werden müssen.