\subsection{Sprint 2}

Im zweiten Sprint gab es zwei Schwerpunkte: Zum einen mussten wir uns langsam Gedanken machen, welche Use Cases wir mit den Google Fusion Tables abdecken wollen. Aus diesen Use Cases sollen dann unabhängige Applikationen entstehen, welche dann das Potential des Produktes aufzeigen sollen. Zum anderen gab es noch ein wichtiges technischen Thema, nämlich die Schreiboperationen. Dazu waren einige Grundlagen von \gls{OAuth} nötig, so dass wir dann die ganze Bandbreite der Schnittstelle nutzen konnten.

Als Nebenthema mussten wir uns noch um den Import von verschiedenen \gls{GIS} Dateien in Fusion Tables kümmern. Zum einen ist dies ein relevantes Thema um eine Migration zu ermöglichen, zum anderen sind viele Daten in beliebigen Formate verfügbar, welche wir natürlich gern als Testdaten nutzen möchten.

Alle Informationen zum Sprint 2 sind auch in unserem Wiki zu finden:
\url{http://redmine.rdmr.ch/redmine/projects/gftprototype/wiki/Sprint_2}

\subsubsection{Hauptaufgaben / Fokussierung im Sprint}
\begin{itemize}
	\item Use Cases erarbeiten
	\item \gls{GIS} Daten in GFT importieren
	\item Informationen über das \emph{Trusted Tester \gls{API}} sammeln
	\item Schreiboperationen (INSERT/UPDATE/DELETE) auf Fusion Tables durchführen können (Authentifizierung mit \gls{OAuth} notwendig)
\end{itemize}

\subsubsection{Ziele}
\begin{itemize}
	\item Finden von Web\gls{GIS} Use Cases (1 "`grosser"' und 2 "`kleine"' Use Cases)
	\item Implementation eines kleinen Use Cases
	\item Geo-Daten importieren und verknüpfen
	\begin{itemize}
		\item \gls{KML} importieren
		\item Converter einsetzen, dann importieren
		\item Verschiedene Fusion Tables joinen/mergen
	\end{itemize}
	\item Trusted Tester \gls{API}
	\begin{itemize}
		\item Zugriff erhalten
		\item \gls{API} testen
	\end{itemize}
\end{itemize}

\subsubsection{Abgabe / Deliverables}
Als ersten umzusetzenden Use Case haben wir uns für \emph{WorldData} (siehe Kapitel \ref{worlddata}) entschieden. Zudem haben wir einen Build-Job (siehe Kapitel \ref{converter-build}) erstellt, welcher es uns erlaubt auf einfache Art und Weise \gls{GIS} Daten zu importieren.

Für \gls{OAuth} haben wir ein Beispiel entwickelt, welches es einem Benutzer ermöglicht seine Tabellen für unsere Applikation freizugeben, so dass dann Schreiboperationen möglich werden.

\begin{itemize}
	\item Erster umgesetzter Use Case \emph{WorldData}
	\item Import-Verfahren für \gls{GIS} Daten
	\item Erweiterte GftLib (Schreiboperationen, Authentifizierung mit \gls{OAuth})
\end{itemize}

\subsubsection{Probleme}
Der Converter-Build war noch nicht sehr ausgereift, so war es noch nicht möglich das Koordinationsystem zu wechseln und überhaupt ein anderes Format zu wählen als GFT.

Bei der Authentifizierung gab es noch einige Probleme um diese für die Library nutzbar zu machen. Für den Fall Google Fusion Tables als Backend-Datenbank zu verwenden und nicht lediglich für den Zugriff auf Fusion Tables eines Benutzers, gab es noch keine Lösung.

Aus zeitlichen Gründen haben wir beim Review des zweiten Sprints zusammen mit dem Betreuer und dem Industriepartner entschlossen, dass wir nur noch einen weiteren "`grossen"' Use Case implementieren werden.