\subsection{Sprint 2}

Im zweiten Sprint gab es 2 Schwerpunkte: zum einen mussten wir uns langsam Gedanken machen, welche Use Cases wir mit den Google Fusion Tables abdecken wollen. Aus diesen Use Cases sollen dann unabhängige Applikationen entstehen, welche so dann das Potential des Produktes aufzeigen sollen. Zum anderen gab es noch ein wichtiges technischen Thema, nämlich die Schreiboperationen. Dazu waren einige Grundlagen von OAuth nötig, so dass wir dann die ganze Bandbreite der Schnittstelle nutzen konnten.

Als Nebenthema mussten wir uns noch um den Import von verschiedenen GIS Dateien in Fusion Tables kümmern. Zum einen ist dies ein sehr relevantes Thema um eine Migration zu ermöglichen, zum anderen sind viele Daten in beliebigen Formate verfügbar, welche wir natürlich gern als Testdaten nutzen möchten.

Alle Informationen zum Sprint 2 sind auch in unserem Wiki zu finden:
\url{http://redmine.rdmr.ch/redmine/projects/gftprototype/wiki/Sprint_2}

\subsubsection{Hauptaufgaben / Fokussierung im Sprint}
\begin{itemize}
	\item Use Cases erarbeiten
	\item GIS Daten in GFT importieren
	\item Informationen über das \emph{Trusted Tester API} sammeln
	\item Schreiboperationen (INSERT/UPDATE/DELETE) auf Fusion Tables durchführen können (Authentifizierung mit OAuth notwendig)
\end{itemize}

\subsubsection{Ziele}
\begin{itemize}
	\item Finden von WebGIS Use Cases (1 \emph{grosser} und 2 \emph{kleine} Use Cases)
	\item Implementation eines kleinen Use Cases
	\item Geo-Daten importieren und verknüpfen
	\begin{itemize}
		\item KML importieren
		\item Converter einsetzen, dann importieren
		\item Verschiedene Fusion Tables joinen/mergen
	\end{itemize}
	\item Trusted Tester API
	\begin{itemize}
		\item Zugriff erhalten
		\item API testen
	\end{itemize}
\end{itemize}

\subsubsection{Abgabe / Deliverables}
Als ersten umzusetzenden Use Case haben wir uns für \emph{WorldData} (Kapitel \ref{worlddata}) entschieden. Zudem haben wir einen Build-Job (Kapitel \ref{converter-build}) erstellt, welcher es uns erlaubt auf einfache Art und Weise GIS Daten zu importieren.

Für OAuth haben wir ein Beispiel entwickelt, welches es einem Benutzer ermöglicht seine Tabellen für unsere Applikation freizugeben, so dass dann Schreiboperationen möglich werden.

\begin{itemize}
	\item Erster umgesetzter Use Case
	\item Import-Verfahren für GIS Daten
	\item Erweiterte GftLib (Schreiboperationen, Authentifizierung mit OAuth)
\end{itemize}

\subsubsection{Probleme}
Der Converter-Build war noch nicht sehr ausgereift, so war es noch nicht möglich das Koordinationsystem zu wechseln und überhaupt ein anderes Format zu wählen als GFT.

Bei der Authentifizierung gab es noch einige Probleme um diese für die Library nutzbar zu machen. Für den Fall, dass wir Google Fusion Tables nicht dazu verwenden möchten auf die Tabellen eines Benutzers zuzugreifen, sondern als Backend-Datenbank zu verwenden, gab es noch keine Lösung.