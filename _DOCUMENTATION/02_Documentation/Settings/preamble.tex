% document type: science report
\documentclass[a4paper, 12pt]{scrreprt}

% font encoding (utf8)
\usepackage[utf8]{inputenc}
\usepackage[T1]{fontenc}

% german hyphenation
\usepackage[ngerman]{babel}

% german literature directory
\usepackage{bibgerm}

% graphics
\usepackage[pdftex]{graphicx}
% wrapping pictures and tables with text
\usepackage{wrapfig}

% define toc formatting
\usepackage[titles]{tocloft}
\setlength{\cftsubsecindent}{3em}
\setlength{\cftsubsecnumwidth}{3.3em}
\setlength{\cftsubsubsecindent}{4.5em}
\setlength{\cftsubsubsecnumwidth}{4em}


% figure numbering
\newcounter{myfigure}
\renewcommand{\thefigure}{\arabic{myfigure}}
\newcounter{mytable}
\renewcommand{\thetable}{\arabic{mytable}}
\usepackage{makeidx} \makeindex
\makeglossary

% Definition vom Seitenlayout
\setlength{\topmargin}{-1.2cm}
\setlength{\oddsidemargin}{0.5cm} 
\setlength{\evensidemargin}{0.5cm}

\setlength{\textheight}{24.5cm} 
\setlength{\textwidth}{15cm}

\setlength{\footskip}{1.2cm} 
\setlength{\footnotesep}{0.4cm}

% Definition vom Header und Footer im Seitenlayout
\usepackage{fancyhdr} 
\pagestyle{fancy} 
\fancyhf{}

% Linien nach dem Header und vor dem Footer
\renewcommand{\footrulewidth}{0.4pt}
\renewcommand{\headrulewidth}{0.4pt}

\fancyhead[L]{\footnotesize{\leftmark}}
\fancyfoot[C]{\footnotesize{\thepage}}
\fancyhead[R]{}

% Header auch bei Kapitelanfangsseite
\def\chapterpagestyle{fancy}

% Neues Kapitel Makro, damit die Variablen korrekt abgefuellt werden
\newcommand{\newchap}[1]{
	\chapter{#1}
	\markboth {Kapitel \thechapter.  {#1}}{Kapitel \thechapter.  {#1}}
}

% Footnote numbering
\newcounter{myfootnote}
\renewcommand{\thefootnote}{\themyfootnote}
\newcommand{\myfootnote}[1]{\stepcounter{myfootnote}\footnote{#1}}

% command to import a figure
\newcommand{\fig}[5]{
  \begin{figure}[h]
    \begin{center}
      \includegraphics[width=#4cm]{#1}
    \end{center}
    \stepcounter{myfigure}
    \caption[#5]{#3}
    \label{#2}
  \end{figure}
}

\newcommand{\figtable}[4]{
  \begin{figure}[h]
    \begin{center}
      {  
			  \footnotesize
  			\sffamily
			  #3
      }
    \end{center}
    \stepcounter{myfigure}
    \caption[#4]{#2}
    \label{#1}
  \end{figure}
}

\newcommand{\tab}[4]{
  \begin{table}[h]
    \begin{center}
      {  
			  \footnotesize
  			\sffamily
  			\renewcommand{\arraystretch}{1.4}
			  #3  			
      }
    \end{center}
    \stepcounter{mytable}
    \caption[#4]{#2}
    \label{#1}
  \end{table}
}

% command to refere to a figure
\newcommand{\reffig}[1]{Abbildung \ref{#1}}

% Definition des Nummerierungslevel
\setcounter{secnumdepth}{4} 
\setcounter{tocdepth}{4} 
\setcounter{lofdepth}{1} 

% Definition von Paragraphen
\parskip=0.3cm
\parindent=0cm

% Definition vom Zeilenabstand
\usepackage{setspace} % Zeilenabstand
\onehalfspacing

% Befehle fuer Anfuehrungszeichen
\usepackage{xspace} % Leerschlag nach Anfuehrungszeichen
\newcommand{\qr}{\grqq\xspace}
\newcommand{\qrs}{\grqq\ }
\newcommand{\ql}{\glqq}

% Formatierungsbefehle zum Zitieren
\newcommand{\page}[1]{S.~#1}
\newcommand{\pagef}[1]{S.~#1f.}
\newcommand{\pageff}[1]{S.~#1ff.}
\newcommand{\pages}[2]{S.~#1--#2}
