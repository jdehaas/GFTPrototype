% document type: science report
\documentclass[abstracton, a4paper, 12pt]{scrreprt}

% Encoding (utf8)
\usepackage[utf8]{inputenc}
\usepackage[T1]{fontenc}

% Silbentrennung (Neu-Deutsch)
\usepackage[ngerman]{babel}

% Literaturverzeichnis (Deutsch)
\usepackage{bibgerm}

% Farben
\usepackage{color}
\usepackage[table]{xcolor}
\definecolor{darkgreen}{rgb}{0,0.6,0}
\definecolor{darkgrey}{rgb}{0.5,0.5,0.5}
\definecolor{grey}{rgb}{0.8,0.8,0.8}
\definecolor{lightgrey}{rgb}{0.95,0.95,0.95}
\definecolor{mauve}{rgb}{0.58,0,0.82}

% Grafiken
\usepackage[pdftex]{graphicx}
\usepackage{epsfig}
% Umfliessen von Text um Tabellen und Bilder
\usepackage{wrapfig}

% Grafiken korrekt positionieren
\usepackage{float}
\restylefloat{figure}
\usepackage[section]{placeins}
\usepackage{subfigure}

% hyperlinks
\usepackage{hyperref}
\hypersetup{
   colorlinks,%
   citecolor=blue,%
   filecolor=blue,%
   linkcolor=blue,%
   urlcolor=blue
}
\urlstyle{same}

% Absatz
\setlength{\parindent}{0pt} % Absatzeinzug
\setlength{\parskip}{10pt} % Absatzabstand

% Glossar
\usepackage[toc]{glossaries}
\makeglossaries

% TODO Kommentare
\usepackage{todonotes}

% Definition vom Header und Footer im Seitenlayout
\usepackage{fancyhdr} 
\pagestyle{fancy} 
\fancyhf{}

% Linien nach dem Header und vor dem Footer
%\renewcommand{\footrulewidth}{0.4pt}
%\renewcommand{\headrulewidth}{0.4pt}

\fancyhead[L]{\footnotesize{\leftmark}}
\fancyfoot[C]{\footnotesize{\thepage}}
\fancyhead[R]{}

% Header auch bei Kapitelanfangsseite
\def\chapterpagestyle{fancy}

% Tabelle
% Padding links und rechts von Zelle
\setlength{\tabcolsep}{5px}
% Padding oben und unten (mittels arraystretch)
\renewcommand{\arraystretch}{1.3}

% Für schöne URLs
\usepackage{url}

% Syntaxhighlighter (benoetigt color und xcolor package)
\usepackage{listings}
 
\lstset{ %
  language=HTML,                % the language of the code
  basicstyle=\footnotesize,           % the size of the fonts that are used for the code
  numbers=left,                   % where to put the line-numbers
  numberstyle=\tiny\color{darkgrey},  % the style that is used for the line-numbers
  stepnumber=1,                   % the step between two line-numbers. If it's 1, each line will be numbered
  numbersep=5pt,                  % how far the line-numbers are from the code
  backgroundcolor=\color{white},  % choose the background color. You must add \usepackage{color}
  showspaces=false,               % show spaces adding particular underscores
  showstringspaces=false,         % underline spaces within strings
  showtabs=false,                 % show tabs within strings adding particular underscores
  frame=single,                   % adds a frame around the code
  rulecolor=\color{darkgrey},        % if not set, the frame-color may be changed on line-breaks within not-black text (e.g. commens (green here))
  tabsize=2,                      % sets default tabsize to 2 spaces
  captionpos=b,                   % sets the caption-position to bottom
  breaklines=true,                % sets automatic line breaking
  breakatwhitespace=false,        % sets if automatic breaks should only happen at whitespace
  title=\lstname,                   % show the filename of files included with \lstinputlisting;
                                  % also try caption instead of title
  keywordstyle=\color{blue},          % keyword style
  commentstyle=\color{darkgreen},       % comment style
  stringstyle=\color{mauve},         % string literal style
  escapeinside={\%*}{*)},            % if you want to add a comment within your code
  morekeywords={*,...}               % if you want to add more keywords to the set
}

% Javascript Syntaxhighliting
\lstdefinelanguage{JavaScript} {
	morekeywords={
		break,const,continue,delete,do,while,export,for,in,function,
		if,else,import,in,instanceOf,label,let,new,return,switch,this,
		throw,try,catch,typeof,var,void,with,yield
	},
	sensitive=false,
	morecomment=[l]{//},
	morecomment=[s]{/*}{*/},
	morestring=[b]",
	morestring=[d]'
}
\lstset{
	frame=tb,
	framesep=5pt,
	basicstyle=\footnotesize\ttfamily,
	showstringspaces=false,
	keywordstyle=\ttfamily\bfseries\color{blue},
	identifierstyle=\ttfamily,
	stringstyle=\ttfamily\color{mauve},
	commentstyle=\color{darkgreen},
	rulecolor=\color{darkgrey},
	xleftmargin=5pt,
	xrightmargin=5pt,
	aboveskip=\bigskipamount,
	belowskip=\bigskipamount
}

%% define toc formatting
%\usepackage[titles]{tocloft}
%\setlength{\cftsubsecindent}{3em}
%\setlength{\cftsubsecnumwidth}{3.3em}
%\setlength{\cftsubsubsecindent}{4.5em}
%\setlength{\cftsubsubsecnumwidth}{4em}

%% figure numbering
%\newcounter{myfigure}
%\renewcommand{\thefigure}{\arabic{myfigure}}
%\newcounter{mytable}
%\renewcommand{\thetable}{\arabic{mytable}}
%\usepackage{makeidx} \makeindex
%\makeglossary

%% Definition vom Seitenlayout
%\setlength{\topmargin}{-1.2cm}
%\setlength{\oddsidemargin}{0.5cm} 
%\setlength{\evensidemargin}{0.5cm}

%\setlength{\textheight}{24.5cm} 
%\setlength{\textwidth}{15cm}

%\setlength{\footskip}{1.2cm} 
%\setlength{\footnotesep}{0.4cm}

%% Neues Kapitel Makro, damit die Variablen korrekt abgefuellt werden
%\newcommand{\newchap}[1]{
%	\chapter{#1}
%	\markboth {Kapitel \thechapter.  {#1}}{Kapitel \thechapter.  {#1}}
%}
%
%% command to import a figure
%\newcommand{\fig}[5]{
%  \begin{figure}[h]
%    \begin{center}
%      \includegraphics[width=#4cm]{#1}
%    \end{center}
%    \stepcounter{myfigure}
%    \caption[#5]{#3}
%    \label{#2}
%  \end{figure}
%}
%
%\newcommand{\figtable}[4]{
%  \begin{figure}[h]
%    \begin{center}
%      {  
%			  \footnotesize
%  			\sffamily
%			  #3
%      }
%    \end{center}
%    \stepcounter{myfigure}
%    \caption[#4]{#2}
%    \label{#1}
%  \end{figure}
%}
%
%\newcommand{\tab}[4]{
%  \begin{table}[h]
%    \begin{center}
%      {  
%			  \footnotesize
%  			\sffamily
%  			\renewcommand{\arraystretch}{1.4}
%			  #3  			
%      }
%    \end{center}
%    \stepcounter{mytable}
%    \caption[#4]{#2}
%    \label{#1}
%  \end{table}
%}
%
%% command to refere to a figure
%\newcommand{\reffig}[1]{Abbildung \ref{#1}}
%
%% Definition des Nummerierungslevel
%\setcounter{secnumdepth}{4} 
%\setcounter{tocdepth}{4} 
%\setcounter{lofdepth}{1} 
%
%% Definition von Paragraphen
%\parskip=0.3cm
%\parindent=0cm
%
%% Definition vom Zeilenabstand
%\usepackage{setspace} % Zeilenabstand
%\onehalfspacing
%
%% Befehle fuer Anfuehrungszeichen
%\usepackage{xspace} % Leerschlag nach Anfuehrungszeichen
%\newcommand{\qr}{\grqq\xspace}
%\newcommand{\qrs}{\grqq\ }
%\newcommand{\ql}{\glqq}
%
%% Formatierungsbefehle zum Zitieren
%\newcommand{\page}[1]{S.~#1}
%\newcommand{\pagef}[1]{S.~#1f.}
%\newcommand{\pageff}[1]{S.~#1ff.}
%\newcommand{\pages}[2]{S.~#1--#2}
