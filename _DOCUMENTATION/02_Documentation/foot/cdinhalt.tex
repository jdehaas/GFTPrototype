\chapter*{Inhalt der CD}
% Titel auch in Kopfzeile anzeigen
\markboth{Inhalt der CD}{Inhalt der CD}
% Kapitel in Inhaltsverzeichnis einfügen
\addcontentsline{toc}{chapter}{Inhalt der CD}

\begin{longtable}{|l|p{10cm}|}
\hline 
\textbf{Verzeichnis} & \textbf{Beschreibung} \\ 
\hline 
\inlinecode{SenchaDesignerProject/} & Beispielapplikation welche mit dem Sencha Designer erstellt wurde \\ 
\hline 
\inlinecode{\_DESIGN/} & Grafik-Rohdaten \\ 
\hline 
\inlinecode{\_DOCUMENTATION/} & Dokumentation der Arbeit \\ 
\hline 
\inlinecode{classes/} & PHP Serverklassen \\ 
\hline 
\inlinecode{examples/} & Beispielapplikationen \\ 
\hline 
\inlinecode{lib/} & Verwendete Library-Pakete \\ 
\hline 
\inlinecode{resources/} & Ressourcen, welche für die Indexseite verwendet wurden (CSS, Bilder) \\ 
\hline 
\inlinecode{services/} & PHP Webservices \\ 
\hline 
\inlinecode{test/} & Tests der UseCases und Libraries \\ 
\hline 
\inlinecode{usecases/} & UseCases welche implementiert wurden (WorldData, FixMyStreet) \\ 
\hline 
\end{longtable} 