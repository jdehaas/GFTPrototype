% Glossar
\newglossaryentry{Geocodierung} {
	name = Geocodierung,
	description = {Bei der Geocodierung wird eine Zeichenkette (Adresse, Namen) einer geografischen Position zugewiesen}
}

\newglossaryentry{KML} {
	name = KML,
	description = {Keyhole Markup Languagen}
}

\newglossaryentry{REST} {
	name = REST,
	description = {Representational State Transfer\cite{rest} ist ein Programmierparadigma, welches besagt, dass sich der Zustand einer Webapplikation als Ressource in Form einer URL beschreiben lässt. Auf eine solche Ressourcen können folgende Befehle angewendet werden: \inlinecode{GET}, \inlinecode{POST}, \inlinecode{PUT},\inlinecode{PATCH}, \inlinecode{DELETE}, \inlinecode{HEAD} und \inlinecode{OPTIONS}. HTTP ist ein Protokoll welches REST implementiert.}
}

\newglossaryentry{AJAX} {
	name = AJAX,
	description = {Asynchronous JavaScript and XML\footnote{\url{http://de.wikipedia.org/wiki/Ajax_(Programmierung)}} beschreibt die Möglichkeit via JavaScript Daten von einem Server nachzuladen.}
}

\newglossaryentry{OAuth} {
	name = OAuth,
	description = {OAuth\footnote{\url{http://de.wikipedia.org/wiki/Oauth}} ist ein offenes Protokoll, das eine standardisierte, sichere API-Autorisierung erlaubt}
}