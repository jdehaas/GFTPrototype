\todo[inline]{Management Summary erstellen}
Das Management Summary soll 2-5 Seiten umfassen sowie eine bis zwei Figuren enthalten. Es richtet sich an den „gebildete Laien“ auf dem Gebiet und beschreibt daher in erster Linie die (neuen und eigenen) Ergebnisse und Resultate der Arbeit. Die Sprache soll knapp, klar und stark untergliedert sein. 
Grundlage für das Management Summary kann der Broschüren-Eintrag sein, den die Abteilung bei Diplomarbeiten jeweils früh verlangt, um eine Broschüre zu drucken. Das Management Summary dient als Vorlage für eine allfällige Web-Publikation.
Das Abstract  und das Management Summary werden - zeitlich gesehen - gegen Schluss der Arbeit geschrieben und bilden zusammen mit den Schlussfolgerungen im technischen Bericht den am häufigsten gelesenen Teil der Arbeit. Diese Dokumente sollen daher am Sorgfältigsten ausgearbeitet sein.
Die folgenden Stichworte sollen die typische Struktur illustrieren, wobei die genaue Ausführung jeweils auf die spezifischen Bedürfnisse und Randbedingungen eines Projekts anzupassen ist. Diese Struktur kann auch für die Präsentation der Arbeit als \emph{Richtschnur} dienen. 
\begin{enumerate}
\item Ausgangslage
	\begin{itemize}
 		\item Warum machen wir das Projekt?
		\item Welche Ziele wurden gesteckt (Kann-Ziele, Muss-Ziele)
		\item Was machen andere / welche ähnlichen Arbeiten gibt es zum Thema?
		\item Vorgehen: Was wurde gemacht? In welchen Teilschritten?
		\item Risiken der Arbeit?
		\item Wer war involviert (Durchführung, Entscheide usw.)?
		\item Was konnte von anderen verwendet werden?
	\end{itemize}
\item Ergebnisse
	\begin{itemize}
 		\item Was ist das Resultat? 
 		\item Bewertung der Resultate, was ist Neuartig an der Arbeit?
 		\item Zielerreichung bezüglich Kann-/Muss-Zielen
 		\item Abweichungen (positiv und negativ) und kurze Begründung dafür (Externe) Kosten der Arbeit?
 		\item Was ist der Nutzen (quantifizierbar/nicht quantifizierbar)?
	\end{itemize}
\item Ausblick
	\begin{itemize}
 		\item Was hat man mit Durchführung des Projekts gelernt?
 		\item Verbleibende Probleme, (zukünftige) Gegenmassnahmen bez. Risiken
 		\item Was würde man anders machen, was ist weiter zu tun
 	\end{itemize}
\end{enumerate}