% Neue Seite beginnen
\cleardoublepage

% Stil des Abstract-Titels veraendern
\renewcommand{\abstractname}{{\Huge\bfseries Abstract}}
% Titel auch in Kopfzeile anzeigen
\markboth{Abstract}{Abstract}

\begin{abstract}
% Kopf- und Fusszeile auch auf Abstractseite
\thispagestyle{scrheadings}

Ziel dieser Arbeit war es, das Potential der \gls{Cloud}-Datenbank \emph{Goolge Fusion Tables} aufzuzeigen. Dazu wurden zuerst einige Beispiele erstellt, welche die verschiedenen Features der Fusion Table verwenden. Zusätzlich wurden Anwendungsfälle im Bereich der Geo-Informationssysteme (\gls{GIS}) gesucht, die sich mit Google Fusion Tables umsetzen lassen würden. Diese wurden dann als Prototypen implementiert, um den Aufwand aufzuzeigen, den es benötigt, um mit Google Fusion Tables und Google Maps eine Web-\gls{GIS}-Applikation zu entwickeln.

Es wurden zwei Anwendungsfälle umgesetzt: Zum einen wurde eine Webapplikation implementiert, die es erlaubt, historische Daten von verschiedenen Ländern miteinander zu vergleichen. Das Ziel dabei war das Anzeigen von grossen Datenmengen auf einer interaktiven Webkarte.

Als zweiten Anwendungsfall wurde eine mobile \gls{WebApp} entwickelt, die es dem Benutzer erlaubt, Defekte – wie zum Beispiel Strassen-Schlaglöcher - in seiner Umgebung an die zuständige Behörde zu melden. Google Fusion Table wurde dabei als Web-Datenbank zur Speicherung der gemeldeten Defekte verwendet.
\end{abstract}