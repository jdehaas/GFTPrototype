\chapter*{Management Summary und Web-Publikation}
\thispagestyle{scrheadings}
% Titel auch in Kopfzeile anzeigen
\markboth{Management Summary und Web-Publikation}{Management Summary und Web-Publikation}

\section*{Ausgangslage}
Mit Google Fusion Tables hat Google einen Dienst lanciert, welcher einerseits eine frei zugängliche Datenbank sein soll und andererseits bei der Visualisierung der Daten helfen soll. Der Fokus liegt dabei auf geografischen Daten, welche dann beispielsweise auf einer Karte angezeigt werden.
Das Ziel dieser Arbeit war es, das Potential von Google Fusion Tables abzuschätzen und dieses anhand von Use Cases aufzuzeigen. Die Use Cases wurden so gewählt, dass ein möglichst breites Spektrum abgedeckt werden konnte. Die Arbeit zeigt mögliche Szenarien auf, um Daten in Google Fusion Tables zu importieren, Daten miteinander zu kombinieren und in geeigneter Form wieder auszugeben. Als weiterer Schwerpunkte wird das Thema Kollaboration behandelt. Dabei geht es darum, dass verschiedene Personen Daten eingeben können,um diese dann als Gesamtes auszuwerten.

Als Rahmenbedingung sollten die Use Cases als Webapplikationen entwickelt werden. Dies ermöglicht einerseits eine grösstmögliche Plattformunabhängigkeit und andererseits zukunftsgerichtete Lösungen. Einer der beiden Use Cases wurde dabei sogar als mobile \gls{WebApp} gestaltet, welche auf allen Smartphones lauffähig ist.

Der Auftrageber dieser Arbeit ist die GEOINFO AG\footnote{\url{www.geoinfo.ch}}, welche massgeschneiderte \gls{GIS}-Softwarelösungen für ihre Kunden entwickelt. Die GEOINFO AG ist sehr interessiert an diesem Thema, um frühzeitig zu erkennen, welche Anwendungsfälle mit Google Fusion Tables abgedeckt werden können und ob diese Dienstleitung allenfalls als Alternative zu einer konventionellen \gls{GIS}-Lösung angesehen werden kann.

Bei dieser Arbeit handelt es sich um eine Studienarbeit welche innerhalb eines Semesters (14 Wochen) an der Hochschule für Technik Rapperswil (HSR) erarbeitet wurde. Insgesamt wurde die Arbeit in 4 Sprints à 3 Wochen aufgeteilt. Der erste Sprint stand ganz im Zeichen des Kennenlernens und der Einarbeitung in das \gls{API}. Im zweiten Sprint wurde ein erster Use Case erarbeitet und im dritten Sprint schliesslich noch ein zweiter. Schliesslich war der letzte Sprint für den Abschluss aller Arbeiten, die Dokumentation und für das Beheben von Fehlern vorgesehen.

\section*{Ergebnisse}
Um die definierten Ziele zu erreichen, wurden zwei Use Cases erarbeitet, welche jeweils mit einer Webapplikation implementiert wurden. Der erste Use Case \emph{WorldData} behandelt das Importieren von Daten in Google Fusion Tables. Für die Ausgabe wurde die FusionTablesLayer-Erweiterung von Google Maps verwendet. Das Ziel hierbei war es vor allem den Umgang mit grösseren Datenmengen sowie  die Kombination verschiedener Datenquellen zu zeigen.

Der zweite Use Case \emph{FixMyStreet} ist eine bekannte Idee aus Bereich der Public Participation \gls{GIS}\footnote{\url{http://en.wikipedia.org/wiki/Public_participation_GIS}}. Dabei sollen Bürger einer Stadt dazu ermutigt werden, Defekte in ihrer Umgebung (Strassenlampen, Schlaglöcher, etc.) zu melden. Die zuständige Behörde kann dann die eingegangenen Meldungen abarbeiten und gegebenenfalls beheben. Um die Handhabung so einfach wie möglich zu machen, wurde dieser Use Case als mobile \gls{WebApp} konzipiert, welcher plattformunabhängig auf allen gängigen Smartphones läuft.

Dank dieser zwei Use Cases und den zahlreichen Beispielanwendungen, welche während der Arbeit erstellt wurden, können wir die Stärken und Schwächen von Google Fusion Tables gut beleuchten. Dank der FusionTablesLayer lassen sich Daten sehr einfach und schnell auf einer Karte darstellen.

Google Fusion Tables als Datenbank für eine Anwendung zu verwenden ist derzeit noch sehr umständlich, gerade auch weil die Dokumentation in diesem Bereich noch unzureichend ist. Wenn jedoch bestehende Daten visualisiert werden sollen, oder jeder Benutzer seine eigenen Tabellen nutzen kann, dann zeigt der Dienst seine wahre Stärke.

\section*{Ausblick}
Die Idee, stukturiert Daten in der \gls{Cloud} zu speichern und auszuwerten, zu hat sicherlich Zukunft. Zum einen ist es für viele Anwendung ausreichend wenn sie eine solche \gls{Cloud}-Datenbank verwenden, so dass nicht selbst eine Infrastruktur aufgebaut werden muss. Es ist derzeit noch gut spürbar, dass es sich bei Google Fusion Tables um ein sehr junges Projekt handelt, welches noch an einigen Stellen unvollständig ist. Schon bald wird ein neues API veröffentlicht, welches wir während dieser Arbeit bereits schon testen konnten. Dies ist sicher ein Schritt in die richtige Richtung.

Für grössere Anwendungen sind derzeit die Limiten nicht akzeptabel. Einerseits lassen sich nur 250MB Daten in einer Tabelle speichern und andererseits werden nur die ersten 100'000 Datensätze bei Abfragen berücksichtigt. Das SQL API bietet zwar Basisfunktionen an, jedoch kommt der Funktionsumfang noch nicht an gängige Datenbanksysteme heran.


